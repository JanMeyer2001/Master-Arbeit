\documentclass[english,version-2020-11]{uzl-thesis}

\UzLThesisSetup{
  Logo-Dateiname        = {Logo_IMI_de.png},
  Verfasst              = {am}{Institut für Medizinische Informatik},
  Titel auf Deutsch     = {MRT-Registration}, 
  Titel auf Englisch    = {MRT-Registration},
  Autor                 = {Jan Meyer},
  Betreuerin            = {Prof. Dr. Mattias Heinrich},
  Mit Unterstützung von = {Ziad Al-Haj Hemidi, Eytan Kats},
  Masterarbeit,
  Studiengang           = {Medizinische Informatik},
  Datum                 = {1. Januar 2025},
  Abstract              = {Something about MRI and registration},
  Zusammenfassung       = {Irgendwas über MRT und Registration},
  Acknowledgements      = {Danke an Mattias, Ziad und Eytan für die gute Betreuung.},
  Numerische Bibliographie %Alphabetische
}


% Designs
%
%
% A \emph{design} is a whole set of font and layout options bundled
% together. They have been chosen in such a way that a visually
% pleasing “overall appearance” results.
%
%
%\UzLStyle{computer modern oldschool design}
%\UzLStyle{computer modern scholary design}
%\UzLStyle{pagella basic design}
%\UzLStyle{pagella centered design}
%\UzLStyle{pagella contrast design}
%\UzLStyle{alegrya basic design}
%\UzLStyle{alegrya scholary design}
%\UzLStyle{alegrya stylish design}
\UzLStyle{alegrya modern design}




%%%%%%%%
%
% Now, include the package you need here using \usepackage. 
%
% However, many standard packages are already loaded by the class:
% amsmath, amssymb, amsthm, babel, biblatex, csquotes, etoolbox,
% filecontents, fontspec, geometry, hyperref, tikz (with libraries
% arrows.meta, positioning and shapes), varioref, url 
%
%%%%%%%




% add bibliography
\addbibresource{Bibliography.bib}
\bibliography{Bibliography.bib}




\begin{document}

\chapter{Introduction}
Introduction of stuff.

\section{Contributions of this Thesis}
We implemented stuff.

\section{Related Work}
Others have already done stuff for example \cite{Chen_2021}, \cite{Jia2023FourierNetLB}, \cite{Wang_deepflash}.

\section{Structure of this Thesis}
This Thesis contains a lot of stuff in different chapters.




\chapter{Basics}

\section{Magnetic Particle Imaging}
Here MRI is described.

\section{Image Registration}
Here goes some stuff for image registration.
\begin{equation}
	T' = \max S(F, T(M))
\end{equation}
This equation is taken from \cite{Chen_2021}

\section{Deep Learning for image registration}
Deep Learning has seen a rise in popularity in the last year in various fields including image registration.

\subsection{CNNs}
CNNs are an important class of neural networks, mainly for image processing.

\subsection{U-Net}
The U-Net architecture is typically used for segmentation, but can also be used for image registration tasks.





\chapter{Main Chapter - Rename later}
In this chapter the main part of the actual work is discussed.

\section{DeepFlash}
Explain DeepFlash.

\section{Fourier Net}
Here goes the explanation for the Fourier-Net.

\section{Fourier Net+}
Here goes the explanation for the Fourier-Net+.




\chapter{Results and Discussion}
Here go the results with the discussion.





\chapter{Conclusion}
Summery of all stuff...

\end{document}